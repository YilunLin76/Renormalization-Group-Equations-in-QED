\documentclass[12pt]{article}
\usepackage{graphicx} % Required for inserting images

\usepackage[utf8]{inputenc}
\usepackage{bm}
\usepackage{cite}
\usepackage{authblk}
\usepackage{slashed}

\usepackage{amsmath, amssymb, mathtools}
\usepackage{tikz}
\usepackage{tikz-feynman}
\usepackage{physics}
\usepackage{braket}
\usepackage{geometry}
\usepackage{setspace}
\usepackage{sectsty}
\usepackage{titling}
\usepackage{tocloft}
\usepackage{hyperref}
\geometry{a4paper,  margin=1in}
\setstretch{1.5}
\numberwithin{equation}{section}
\AtBeginDocument{\Large \selectfont}

\author{YILUN LIN}
\date{October 2025}
\title{Renormalization Group Equations in Quantum Electrodynamics\\[5pt]{\large A Detailed Technical Note}}
\renewcommand{\contentsname}{\centering\Large\bfseries Table of Contents}
\begin{document}
\begin{titlepage}
    \maketitle
\end{titlepage}
\newpage
\setcounter{page}{2}

\begin{center}
\tableofcontents
\end{center}
\thispagestyle{plain}

\clearpage
\newpage

\newpage

\section{Introduction}
Quantum Electrodynamics (QED) is a perturbatively well-defined quantum field theory, yet its perturbative expansion necessarily produces ultraviolet (UV) divergences in loop amplitudes. These divergences arise from integrations over arbitrarily high virtual momenta in Feynman graphs and reflect the fact that QED, viewed as a continuum field theory, possesses no intrinsic physical cutoff. Renormalization is therefore required in order to define finite Green's functions and physical observables.
\newline 
From a physical standpoint, renormalization reflects the fact that the parameters appearing in the bare Lagrangean (i.e. bare charge $e_{0}$, bare mass $m_{0}$, and the bare photon field normalization) are not directly measurable quantities. Instead, measurements probe effective parameters defined at a particular energy scale. The renormalization procedure systematically relates these scale-dependent effective parameters to the divergent bare parameters in such a way that all S-matrix elements remain finite.
\newline
The central observation is that divergences do not imply inconsistency. Rather, they indicate the need to reinterpret the parameters of the theory. The renormalization group (RG) then captures how these effective parameters evolve with changes of the renormalization scale. In QED, this scale dependence is physically meaningful that the running of the effective charge $e\left(\mu\right)$ describes the phenomenon of charge screening by virtual electron-positron pairs.

\section{QED Lagrangean and Renormalization Setup}
\subsection{Bare and renormalized quantitties}
We start with the bare Lagrangean:
\begin{equation}
    \begin{split}
        \mathcal{L}_{0}\,=\,-\frac{1}{4}F_{0\mu\nu}F_{0}^{\mu\nu}+\bar{\psi}_{0}\left(i\slashed{\partial}-m_{0}\right)\psi_{0}-e_{0}\bar{\psi}_{0}\gamma^{\mu}\psi_{0}A_{0\mu},
    \end{split}
\end{equation}
where $\slashed{\partial}\,=\,\gamma^{\mu}\partial_{\mu}$.
\newline Renormalization is implemented by expressing bare fields and parameters in terms of renormalized ones and their corresponding renormalization constants:
\begin{align}
    \psi_{0} &= Z_{2}^{1/2}\psi, \\
    A_{0\mu} &= Z_{3}^{1/2}A_{\mu}, \\
    m_{0} &= Z_{m}m, \\
    e_{0} &= Z_{e}e.
\end{align}
Gauge invariance implies the Ward identity
\begin{equation}
    \begin{split}
        Z_{1}\,=\,Z_{2},
    \end{split}
\end{equation}
where $Z_{1}$ is the vertex renormalization constant. This reduces the number of independent renormalization constants and guarantees charge renormalization is controlled entirely by the photon field renormalization
\begin{equation}
    \begin{split}
        Z_{e}\,=\,Z_{3}^{-1/2}.
    \end{split}
\end{equation}
Thus, the renormalized QED Lagrangean becomes
\begin{equation}
    \begin{split}
        \mathcal{L}\,=\,&-\frac{1}{4}Z_{3}F_{\mu\nu}F^{\mu\nu}\,+\,Z_{2}\bar{\psi}i\gamma^{\mu}\partial_{\mu}\psi\, \\
        &-\,Z_{m}m\bar{\psi}\psi\,+\,Z_{1}e\bar{\psi}\gamma^{\mu}\psi A_{\mu}.
    \end{split}
\end{equation}
The renormalization constants $Z_{i}$ are computed perturbatively by imposing renormalization conditions. In dimensional regularization with minimal subtraction, these renormalization constants take the form
\begin{equation}
    \begin{split}
        Z_{i}\,=\,1+\Sigma_{k\geq1}\frac{a_{i}^{\left(k\right)}\left(e\right)}{\epsilon^{k}},
    \end{split}
\end{equation}
where $\epsilon\,=\,4-d$.

\section{One-Loop Corrections in QED}
\subsection{Vacuum Polarization}
The one-loop photon self-energy diagram (electron loop) contributes:
\begin{equation}
    \begin{split}
        \Pi^{\mu\nu}\left(q\right)\,=\,\left(q^{\mu}q^{\nu}\,-\,q^{2}g^{\mu\nu}\right)\Pi\left(q^{2}\right),
    \end{split}
\end{equation}
where
\begin{equation}
    \begin{split}
        \Pi\left(q^{2}\right)\,=\,\frac{e^{2}}{12\pi^{2}}\left(\frac{2}{\epsilon}+\ln\frac{\mu^{2}}{m^{2}}+\cdots\right).
    \end{split}
\end{equation}
Thus
\begin{equation}
    \begin{split}
        Z_{3}\,=\,1-\frac{e^{2}}{6\pi^{2}}\frac{1}{\epsilon}.
    \end{split}
\end{equation}

\begin{center}
\begin{tikzpicture}
   \begin{feynman}
       \vertex (L) at (-3.0,0) {};
       \vertex (R) at (3.0,0) {};
       \vertex (a) at (-1.0,0) {};
       \vertex (b) at (1.0,0) {};
       \vertex (top) at (0,1.2) {};
       \vertex (bot) at (0,-1.2) {};
       \diagram*{
         (L) -- [photon, edge label=\(p\)] (a),
         (a) -- [fermion, half left, looseness=1.6, edge label=\(k\)] (b),
         (b) -- [fermion, half left, looseness=1.6, edge label'=\(k-p\)] (a),
         (b) -- [photon, edge label=\(p\)] (R),
       };
   \end{feynman}
\end{tikzpicture}
\newline
Figure: Vacuum polarization in QED.
\end{center}
The mathematical expression of the above Feynman diagram is
\begin{equation}
    \begin{split}
        \left(-ie\right)^{2}\int\!\frac{\mathrm{d}^{4}k}{\left(2\pi\right)^{4}}\frac{i\left(2k^{\mu}-p^{\mu}\right)}{\left(k-p\right)^{2}-m^{2}+i\epsilon}\frac{i\left(2k^{\nu}-p^{\nu}\right)}{k^{2}-m^{2}+i\epsilon}.
    \end{split}
\end{equation}

\subsection{Vertex correction}
By Ward identity:
\begin{equation}
    \begin{split}
        Z_{1}\,=\,Z_{2}.
    \end{split}
\end{equation}

\section{Renormalization Group and Running Coupling}
\subsection{General form of Callan--Symanzik equation}
\begin{equation}
    \begin{split}
        \left(\mu\frac{\partial}{\partial \mu}\,+\,\beta\left(e\right)\frac{\partial}{\partial e}\,-\,n_{\psi}\gamma_{\psi}\,-\,n_{A}\gamma_{A}\right)\Gamma\,=\,0.
    \end{split}
\end{equation}

\subsection{QED beta function}
Since $Z_{e}\,=\,Z_{3}^{-1/2}$,
\begin{equation}
    \begin{split}
        \beta\left(e\right)\,&=\,\mu\frac{\mathrm{d}e}{\mathrm{d}\mu} \\
        &=\,\frac{e^3}{12\pi^2}\,+\,\mathcal{O}\left(e^5\right).
    \end{split}
\end{equation}

\subsection{Running coupling}
Define $\alpha\,=\,\frac{e^{2}}{4\pi}$:
\begin{equation}
    \begin{split}
        \alpha\left(\mu\right)\,\coloneqq\,\frac{\alpha\left(\mu_{0}\right)}{1-\frac{\alpha\left(\mu_{0}\right)}{3\pi}\ln\left(\mu/\mu_{0}\right)}.
    \end{split}
\end{equation}

\section{Anomalous Dimensions}

\begin{align}
    \gamma_{\psi}\,&=\,\frac{e^{2}}{8\pi^{2}}, \\
    \gamma_{A}\,&=\,-\frac{e^{2}}{6\pi^{2}}.
\end{align}

\section{Physical Interpretation}
Vacuum polarization causes charge screening, making the effective electric charge scale-dependent. At higher energies, the electron cloud is probed more deeply, revealing a larger bare charge.

\section{Landau Pole and Triviality}
The running coupling diverges at the Landau pole:
\begin{equation}
    \begin{split}
        \mu_{\rm LP}\,=\,\mu_{0}\exp\left(\frac{3\pi}{\alpha\left(\mu_{0}\right)}\right).
    \end{split}
\end{equation}
This suggests QED is an effective field theory valid below extremely high scales.

\section{Summary}

We derived the one-loop beta function and anomalous dimensions in QED, obtained the running coupling, and discussed physical consequences such as charge screening and the Landau pole.

\end{document}
